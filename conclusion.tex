\chapter{Conclusion}
In this section of conclusion, an overview is presented which reflects on this project, how it could have been improved upon, personal reflections, contributions, achievements acquired, and the future work which could be implemented into the project.

\section{Personal Reflection}
\subsection{Cathal Butler}
After closing the book on the code for the project and dissertation I am very happy with what has been accomplished. I started out with Dart being a band new language to me and only knowing a very small amount about WebRTC, to know having a pretty good understand of both of them. Learning Flutter on top of Dart was definitely a learning curve with it use of widgets and Object-oriented programming, but once understood is a very nice framework that can be used to design beautiful and very functional applications. I feel every aspect of this project thought me something new and challenged me so looking back at it I wouldn't change anything about the project specification. Changes I would make is the scope of the project. I feel if assigned work was complete on time the application would not have fell down where it did I feel it would have allowed adequate time to debug, research and implement solutions.
\\\\ I will be taking a lot away with me from this project when it comes to research, planning and implementation. I feel this project has defiantly shown me what a real world project is like and the importance of communication with a team, version control, correct coding guidelines and documentation.
\\\\ Overall I would like to thank my teammates for the work they have put to get the project to were it is today, I feel as a team we have constructed a very capable project. 
\\\\ Finally I would like to thank my lectures in college mainly John French our supervisor for his support, advice and time to help us to do the best we could.    



\subsection{David Neilan}
\subsection{Morgan Reilly}
Looking back on the project there were definitely some things I would have done differently. I think for the most part the topic chosen for the project was adequately challenging, if not quite difficult, but was not something I found myself really getting into. This, I feel, was attributed to Flutter as a front-end technology. The reason being is that with the time-frame of this project, the other projects for our other modules, and my work-life balance which I was trying to juggle, that it was quite difficult to learn and become fluent in Flutter in the time frame given. This in itself caused blocking issues throughout the duration of the project. Flutter in itself is a decent programming language, but it is not one which I see myself revisiting. In terms of the overall choice of topic and frameworks, I think it could have been scoped with more scrutiny. We had an idea for quite some time but didn’t make up our mind on certain aspects until quite late in the project, which also caused a few headaches. Ultimately though it made for quite a challenging project which was outside the norm of what I may have undertaken should I have been on my own.
\\\\ If I had my time over again, and had to re-do this project, I would have gone solo. The reason being is that should any issues have arisen, they would have been through my own fault and would have been met with more active decision. I also would have gone with a topic that would have been more AI related as I intend to pursue a post-grad in that area. That being said I would like to stress that both of my other 2 team mates were terrific to work with, despite some bumpy roads along the way, and are both exceptionally talented and focused in the work they needed to do. Some deadlines definitely could have been met sooner, but we ultimately got a proof of concept working for WebRTC with Flutter, and although the application is not what it could have been, I’m definitely proud of what we achieved as a team. 
\\\\ Overall I am quite happy with how this project turned out. Despite its short comings it was approached from a very professional manner, and our goal was ultimately hit in the end. I would like to thank our supervisor, John French, for his patience and advice with us along the way, and also like to thank my teammates for pulling together and finalising this piece of work. 
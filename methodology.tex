\chapter{Methodology}
\section{Project Scope \& Goal}
In terms of project scope we needed to aim for something ideally with quite a few moving parts, with use of as much new technology as possible. Initial ideas consisted of the likes of machine learning or the use of computer vision. We looked at possible academic solutions, with emphasis on security and encryption. One idea was a web store for a local farmers market, which with three people in the group, was declined as there would be a possible factor of work-load starvation between us. 
\\ Based upon those initial concepts we toyed with the idea of an agricultural application which would use a mixture of drone monitoring, computer vision and artificial intelligence to track the location and behaviour of cattle for farms. This idea was scrapped due to a lack of relevant data sets, and an un-achievable goal.  From that a data driven statistics project was examined and also subsequently scrapped. 
\\ Ultimately we settled on a self hosted conference call application, one which we had intended to base on Discord, and draw inspiration from the likes of Team Speak / Teams / Skype. The initial plans consisted of an application which would be self-hosted, and containerised, and which would hinge on the use of the relatively new front-end technology, Flutter. Through the use of Flutter, we would be able to have an array of platforms on which to deploy to, such as mobile or desktop. At this point in the project we were still un-decided on a back-end technology.
\\ First steps involved building a feature list from various other sites and applications, i.e. draw inspiration from the design of Discords server/chat rooms, and verifying WebRTC compatibility with Flutter. It also involved arranging weekly meetings, both with our supervisor, but also amongst ourselves with the intent to set our weekly goals and to re-convene to check-in on our progress of the project. 
\\ Our initial scope consisted of building a conference application through Flutter. It would use WebRTC as it’s main communication technology rather than Voice Over IP (VOIP). It would have multimedia storage, such as images/videos, but would primarily focus on voice/video calling and multi-peer text messaging. It would also support Oauth as a form of authentication/login. 

\section{Development Methods}

\section{Project \& Time Management}
At an early stage in the project we opted to use an issue tracking software to manage the workload, allocation of tasks, and to monitor our current stage of development.
Initially we looked at Jira, by Atlassian, which was unfortunately a bit costly and had no education deals. Upon further digging we found YouTrack, which has the same ticket issuing capabilities as Jira, but was free for student use.
\subsection{YouTrack}
\subsection{Jenkins}
\section{Issues Encountered}